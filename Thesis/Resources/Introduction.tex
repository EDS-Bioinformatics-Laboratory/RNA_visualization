\chapter{Introduction}
\lhead{\emph{Introduction}}
% \subsection{Gene-expression analysis and dimensionality reduction}
% \subsubsection{Gene-expression}
Gene expression is the process where genes on the DNA are transcribed and used to create specific products. It is the process that sets the synthesis of proteins from the DNA in motion and it is fundamental to all biological mechanisms. When a gene is expressed, its DNA sequence is copied to a piece of mRNA, which can then be translated into a protein. Genes that are highly expressed are being transcribed to mRNA more often than genes that are lowly expressed, so the number of mRNA transcripts for these genes is relatively high. Given the biological and medical significance of this process, it is studied broadly and techniques for quantitative measurement have been developed over the last decades.
RNA sequencing (\textbf{RNA-seq}) measures the relative level of expression of genes by quantifying the number of mRNA transcripts. This approach measures the average levels of gene expression across a population of cells, which possibly consists of different cell types. This is a limitation to many experiments because results are not specific to one individual cell.
Single-cell RNA-sequencing (scRNA-seq) \cite{haque2017practical} has given the means to analyse the expression patterns of large numbers of genes from specifically chosen cells. 
First, a sample of specifically chosen cells is collected. The cells undergo a lysis step that permeates their membrane but leaves the RNA intact. The mRNA is then reverse-transcribed to its complementary cDNA. The newly acquired cDNA is amplified, through polymerase chain reaction (PCR) or in vitro amplification. Finally, this cDNA is sequenced and the resulting sequence reads are mapped back to the genome, pieces of cDNA that match with a gene are counted.

This technique has seen a rise in popularity \cite{shapiro2013single}. ScRNA-seq was first pioneered in the late 2000's \cite{tang2009mrna}. %\cite{weber2015discovering, nagalakshmi2008transcriptional}. 
The new invention of scRNA-seq allowed researchers to study individual cell types and to compare different cell types. For example, cells from different embryonic stages might be compared to study the activity of genes during embryonic development \cite{deng2014single}. Another example would be the detection of malignant tumor cells in otherwise healthy cell populations \cite{tirosh2016dissecting}.


%When RNA-seq first started out, the RNA of one cell from one species was analyzed. This allowed the observers to write evaluations on the transcriptome of one species. Since then, the technology has evolved enough to allow the analysis of the transcriptome of thousands of cells. One advantage of this is that, now, multiple groups of different cells can be studied and compared. For example, it is possible to compare cells from different body parts or in different health conditions. Newer technologies have even been invented to enable the comparison of different species, despite their different genomes \cite{aubry2014deep}. However, comparing multiple groups of cells did require new forms of analysis. Seeing as how samples contain the expression levels of a large number of genes, it may be tedious to compare each and every of those genes for different groups. Moreover, it is hard to summarize the findings on one of such an experiment into one visualization plot that represents most or all of the information obtained through the experiment. 

% \subsubsection{Dimensionality reduction for visualization purposes}
For the comparison of transcriptomes of different cell types, it is often desirable to visualize the gene expression levels of individual cells in a simple yet informative way.
It may not be hard to visualize data in two dimensions, for instance, one could generate a scatter-plot of two genes, where each point corresponds to one cell. Simple correlations should then easily become visible. For three dimensions or genes, a 3D plot can be generated similarly. For data of more dimensions, it is possible to produce multiple plots for every combination of two dimensions. However, scRNA-seq deals with data that may exceed thousands of genes, which is hard to visualize in a way that preserves most information. Dimensionality reduction has therefore become an important topic in scRNA-seq data analysis. Techniques like principal component analysis (\textbf{PCA}) \cite{pearson1901liii}, probabilistic PCA (\textbf{PPCA}) \cite{ppca}, t-distributed stochastic neighbour embedding (\textbf{t-SNE}) \cite{maaten2008visualizing} and uniform manifold approximation and projection (\textbf{UMAP}) \cite{mcinnes2018umap, becht2019dimensionality} (among others) deal with dimensionality reduction and try to transform high-dimensional data to 2 or 3 dimensional data in the most adequate ways. For instance, PCA projects the data onto lesser dimensions, known as the principal components, in such a manner that the variance of the projected data is kept at a maximum. PPCA attempts to find latent random variables that explain the observed data in a probabilistic way. These techniques are linear. T-SNE and UMAP are non-linear techniques. These techniques put the data-points in the low dimensional space by specifying an attractive force between points that are closely related in the data space and a repellent force between other points.

Another way to induce non-linearity is by adding hierarchical layers. For example, a hierarchical mixture of PPCAs (\textbf{HmPPCAs}) \cite{bishop1998hierarchical} is able to divide the data into different clusters and returns their individual low-dimensional representation. It has been argued that each of these clusters might have different optimal representations in the low-dimensional space. While one cluster can be best described by two principal components, another cluster might be better represented by two different principal components. Representing every cluster by the same principal components might be optimal for the whole data-set, but not for the separate clusters. This model has been used for the analysis of scRNA-seq data \cite{philipsthesis}. However, this model was introduced as an interactive model, which still requires the user to select the number of different clusters and their approximate locations in the low-dimensional space. User input may be susceptible to human subjectivity and inconsistency. It also requires the user to wait for the parameter estimation to finish so that they can initiate the next level. Here, we develop a model that works in a fully automated way and does not require extra user-input.

Another difference to our model is how it is solved. Some of the mentioned techniques, like PCA and PPCA, have a closed-form solution. Alternatively, it is possible to use an Expectation-Maximization (\textbf{EM}) algorithm (see Appendix \ref{app:gmm_em}), which is guaranteed to converge to a local optimum, with lower computational cost than the closed-form solution. The HmPPCAs model as proposed originally was solved using an EM-algorithm as well \cite{bishop1998hierarchical, philipsthesis}. However, another approach to find a solution to the HmPPCAs is through probabilistic programming. Probabilistic programming entails programming a probabilistic model and having the underlying distributions of the parameters found automatically. Platforms for statistical optimization, like TensorFlow Probability \cite{dillon2017tensorflow} and Stan \cite{carpenter2017stan} have been released as an alternative way to approach solutions to statistical problems. These are Bayesian probabilistic programming languages. Using tools like these has the benefit that the user can specify and manipulate statistical models without having to derive and solve the necessary equations. This increases the range of models to be studied and enables the user to easily refine parts of the model without having to rewrite the entire solution. These tools do not only give the users a solution to their model, but they can also give more information about the solution as they compute (an approximation of) the entire probability function of the parameters of the model given the input data. The models created in this report will be created within the Stan environment using Python as the primary programming language. Stan is able to automate the inference of probabilistic models in two ways. The first is by sampling through the No U-Turn Sampler (\textbf{NUTS}), which is an extension to the Hamiltonian Monte Carlo method. The other is an algorithm called automatic differentiation variational inference (\textbf{ADVI}). Both these methods will be discussed in Section \ref{sec:stan}

% Models like these have been the status quo for scRNA-seq visualization and a variety of other purposes. But even though these models hold large value to gene-expression analysis, they are not tuned specifically for this function. More recently, some models have been developed that are tailored specifically for gene-expression analysis  \cite{zifa}. Next to that, many of those models are expanded to find more and more complex structures in data generally  \cite{bishop2002bayesian}, which could also qualitatively improve expression-pattern analysis. The objective of these writings is to produce a dimensionality reduction model, suited specifically for the purpose of gene-expression data.

% To achieve this, we will attempt to incorporate a range of technical developments in the field to the existing PPCA model.\todo{whats new in this model? difference from hmppca} The model can then be compared with other models\todo{which models?} 

Since we will mainly extend the HmPPCAs model, Section \ref{sec:approach} starts with the theory behind PPCA and hmPPCAs models. We then continue by discussing Stan and the sampling methods used to solve our models. We then conclude this section with a description of the algorithm used for this project. Section \ref{sec:methods} describes the setup of the model and the experiments to evaluate our model and to compare it with other models. The results of these experiments will be discussed in Section \ref{sec:results}, and we conclude this report with a brief discussion in Section \ref{sec:discussion}.